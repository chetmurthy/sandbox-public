\documentclass[a4paper,12pt]{article}

\usepackage[utf8]{inputenc}
\usepackage[french]{babel}
\usepackage{fullpage}
\usepackage[hidelinks]{hyperref}

\title{De l'usage de \texttt{break}, \texttt{continue} et \texttt{return}}
\author{\href{https://www.lri.fr/~filliatr/index.fr.html}{Jean-Christophe Filliâtre}}

\newcommand{\bcr}{\texttt{break}/\texttt{continue}/\texttt{return}}

\begin{document}

\maketitle

\begin{abstract}
  On rencontre régulièrement des enseignants qui se refusent
  catégoriquement à utiliser les constructions \texttt{break} et
  \texttt{continue} ou encore à sortir d'une boucle avec une
  construction \texttt{return}. C'est bien dommage. Je donne ici des
  arguments allant au contraire en faveur de l'utilisation de ces
  trois constructions.
\end{abstract}

% Je vais vous donner mon avis très volontiers. Il n'est que mon avis
% personnel, bien entendu, mais je vais essayer d'argumenter autant que
% possible. J'ôte de suite tout suspens : je suis pour les sorties anticipées.

Tout ce que je vais dire ci-dessous s'applique autant à la sortie
anticipée de boucle avec \texttt{break}, qu'à l'utilisation de
\texttt{continue} et qu'à la sortie anticipée de fonction avec
\texttt{return} (y compris depuis l'intérieur d'une boucle).  Voici
mes arguments en faveur de l'utilisation de ces trois constructions.
\begin{enumerate}
\item Quand on utilise un langage, et surtout quand on l'enseigne, il
  faut le faire \textbf{idiomatiquement}. C'est vraiment desservir les étudiants
  que de ne pas leur montrer la façon idiomatique de faire, car c'est
  celle qu'ils trouveront dans la bibliothèque standard et dans le
  code écrit par d'autres, mais aussi parce que c'est probablement
  celle qui est compilée le plus efficacement (sinon, elle ne serait
  pas l'idiome).

\item Se passer de \bcr\ quand on écrit des boucles oblige à
  des \textbf{contorsions}, à base de variables booléennes le plus souvent, qui
  \begin{itemize}
  \item obscurcissent le code, pour le programmeur comme pour le
    lecteur ;
  \item augmentent considérablement le risque d'erreur.
  \end{itemize}
  Au-delà de ces deux phénomènes, le code est généralement plus long, ce
  qui est regrettable.

\item Dans le même esprit, ces trois constructions permettent de
  \textbf{simplifier le flot de contrôle en le conservant le plus linéaire
  possible}. Ainsi, un \texttt{return} permet de se passer d'un
  \texttt{else} :
\begin{verbatim}
  def f(x):
    if x == 0: return 1
    ...
\end{verbatim}
  (J'utilise ici Python à des fins d'illustrations mais
  le langage importe peu.)
  Nul besoin d'indenter tout ce qui vient après ce \texttt{return}, ce qui est
  d'autant appréciable que le reste de la fonction est gros. Notez que
  c'est la même chose avec une exception :
\begin{verbatim}
  def f(x):
    if x < 0: raise ValueError
    ...
\end{verbatim}
  Il en va de même pour un \texttt{continue}. Ainsi, il est agréable d'écrire
\begin{verbatim}
  while len(q) > 0:
    x = q.pop()
    if x in vus: continue
    ...
\end{verbatim}
  plutôt que d'utiliser tout un nouveau bloc pour y mettre le contenu de
  \texttt{...}, qui peut être arbitrairement gros.

\item Enfin, les constructions \bcr\ ont une
  \textbf{sémantique simple}. (Quand on s'intéresse à la compilation, on
  comprend rapidement que ce sont mêmes là les constructions les plus
  simples à compiler : un simple saut vers un endroit statiquement
  connu, facilement identifié.)  Les langages de programmation
  contiennent tous de très nombreuses subtilités et je comprends qu'un
  enseignant puisse délibérément écarter les aspects les plus sordides
  (je le fais moi-même quand j'enseigne à des débutants). Mais pourquoi
  écarter des constructions simples, qui rendent de plus le code plus
  élégant ?
\end{enumerate}

\paragraph{À propos des langages fonctionnels.}
Si les constructions \bcr\
sont présentes dans la plupart des langages impératifs, on constate
généralement leur absence dans les langages fonctionnels comme
Haskell, OCaml, Standard ML, F\#, etc.

Ceci peut s'expliquer techniquement par le fait que, dans les langages
fonctionnels, il est courant d'utiliser une fonction récursive plutôt
qu'une boucle. Dès lors, la sortie anticipée est immédiate : il suffit
de ne pas faire d'appel récursif. Ainsi, en OCaml, je peux chercher
l'indice du premier 0 dans un tableau avec cette fonction récursive :
\begin{verbatim}
  let rec cherche a i =
    if i = Array.length n then raise Not_found;
    if a.(i) = 0 then i else cherche a (i + 1) in
  cherche a 0
\end{verbatim}
Les langages fonctionnels optimisent typiquement l'appel terminal (c'est
le cas d'OCaml), ce qui fait que cette fonction \texttt{cherche} va se
retrouver compilée \textbf{exactement} comme la boucle
\begin{verbatim}
  i = 0
  while i < len(a):
    if a[i] == 0: return i
    i += 1
  raise NotFound
\end{verbatim}
qui contient une sortie anticipée. Dit autrement, les langages
fonctionnels n'ont pas besoin de telles constructions. (Cela étant,
j'adorerais disposer de
\bcr\ en OCaml !) Dans les
langages impératifs, en revanche, il n'est pas usuel de recourir à des
fonctions récursives à la place de boucle et surtout (peut-être pour
cette raison) les appels terminaux sont rarement optimisés (jamais en
Java et Python, par exemple), ce qui expose au débordement de pile.

\bigskip
\noindent \emph{Merci à Alexandre Casamayou de
  m'avoir amené à rédiger et publier cette note.}

\end{document}

% Local Variables:
% compile-command: "rubber -d bcr.tex"
% ispell-local-dictionary: "francais"
% End:
